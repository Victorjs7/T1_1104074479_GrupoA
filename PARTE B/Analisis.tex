\documentclass[12pt]{report}
%%\usepackage[utf8]{inputenc}
\usepackage[spanish]{babel}
\usepackage[utf8]{inputenc} 
\usepackage{amsmath,amsthm,amsfonts,amssymb}
\usepackage{Sweave}
\usepackage{graphicx}
\usepackage{hyperref}
\usepackage{anysize}
\marginsize{1.78cm}{1.65cm}{1.78cm}{1.78cm} 

\title{\Huge Universidad Nacional de Loja \\ Ingeniería en Sistemas \\ Análisis y Diseño de Sistemas}

\author{\includegraphics[width=9cm, height=7cm]{logo-unl.jpg} \\\\ {Victor Francisco Jumbo Sinchire} \\\\ \texttt{vfjumbos@unl.edu.ec}}

\begin{document}
\Sconcordance{concordance:Analisis.tex:Analisis.Rnw:%
1 39 1 1 2 41 0 1 2 2 1 1 2 11 0 1 2 4 1}

\maketitle

\begin{center}\textbf{CUESTIONARIO DE PREGUNTAS}\end{center}
\textbf{¿Es aplicable la ingeniería de software cuando se elaboran webapps? Si es así,¿cómo puede modificarse para que asimile las características únicas de éstas?}\\
Si es posible aunque sea un enfoque diferente si se  puede ya  que los sistemas y aplicaciones basados en web involucran una mezcla entre las publicaciones impresas y el desarrollo de software, etc. Y nos basamos en las siguientes características:
\begin{itemize}
\item Uso intensivo de redes.
\item Concurrencia.
\item Carga impredecible.
\item Rendimiento
\item Disponibilidad.
\item	Orientadas a los datos-Contenido sensible.
\item	Evolución continúa.
\item	Seguridad.
\end{itemize}

\begin{center}\textbf{RESOLUCIÓN DEL DATASET Titanic}\end{center}
\textbf{Descripción}

Este conjunto de datos proporciona información sobre el destino de los pasajeros en el primer viaje fatal del trasatlántico Titanic, que se resumen de acuerdo a la situación económica (clase ), el sexo, la edad y la supervivencia.
\\\\

\textbf{Contenido del Dataset Titanic}
\begin{Schunk}
\begin{Sinput}
> Titanic
\end{Sinput}
\begin{Soutput}
, , Age = Child, Survived = No

      Sex
Class  Male Female
  1st     0      0
  2nd     0      0
  3rd    35     17
  Crew    0      0

, , Age = Adult, Survived = No

      Sex
Class  Male Female
  1st   118      4
  2nd   154     13
  3rd   387     89
  Crew  670      3

, , Age = Child, Survived = Yes

      Sex
Class  Male Female
  1st     5      1
  2nd    11     13
  3rd    13     14
  Crew    0      0

, , Age = Adult, Survived = Yes

      Sex
Class  Male Female
  1st    57    140
  2nd    14     80
  3rd    75     76
  Crew  192     20
\end{Soutput}
\end{Schunk}

\textbf{¿Cuál es el número total de casos en el dataset?}

\begin{Schunk}
\begin{Sinput}
> summary(Titanic)
\end{Sinput}
\begin{Soutput}
Number of cases in table: 2201 
Number of factors: 4 
Test for independence of all factors:
	Chisq = 1637.4, df = 25, p-value = 0
	Chi-squared approximation may be incorrect
\end{Soutput}
\end{Schunk}

\begin{center}
\href{http://creativecommons.org/licenses/by-nc/4.0/}{\includegraphics[width=4cm, height=1.5cm]{cc.png}}
\end{center}
\end{document}
